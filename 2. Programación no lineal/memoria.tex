\documentclass[12pt,a4paper,twoside,openright,titlepage,final]{article}
\usepackage{fontspec}
\usepackage{amsmath}
\usepackage{amsfonts}
\usepackage{amssymb}
\usepackage{makeidx}
\usepackage{graphicx}
\usepackage[hidelinks,unicode=true]{hyperref}
\usepackage[spanish,es-nodecimaldot,es-lcroman,es-tabla,es-noshorthands]{babel}
\usepackage[left=3cm,right=2cm, bottom=4cm]{geometry}
\usepackage{natbib}
\usepackage{microtype}
\usepackage{ifdraft}
\usepackage{verbatim}
\usepackage[nottoc]{tocbibind}
\usepackage{pdflscape}
\usepackage{fancyvrb}
\usepackage[obeyDraft]{todonotes}
\ifdraft{
	\usepackage{draftwatermark}
	\SetWatermarkText{BORRADOR}
	\SetWatermarkScale{0.7}
	\SetWatermarkColor{red}
}{}
\usepackage{booktabs}
\usepackage{longtable}
\usepackage{calc}
\usepackage{array}
\usepackage{caption}
\usepackage{subfigure}
\usepackage{footnote}
\usepackage{url}
\usepackage[titletoc]{appendix}

\setsansfont[Ligatures=TeX]{texgyreadventor}
\setmainfont[Ligatures=TeX]{texgyrepagella}
\setmonofont{FreeMono}

\usetikzlibrary{decorations.pathreplacing}

\input{portada}

\author{José Ignacio Escribano}

\title{}

\setlength{\parindent}{0pt}

\begin{document}

\pagenumbering{alph}
\setcounter{page}{1}

\portada{Caso Práctico II}{Gestión y planificación}{Optimización no lineal}{José Ignacio Escribano}{Móstoles}

\listoffigures
\thispagestyle{empty}
\newpage

\listoftables
\thispagestyle{empty}
\newpage

\tableofcontents
\thispagestyle{empty}
\newpage


\pagenumbering{arabic}
\setcounter{page}{1}

\section{Introducción}



\section{Resolución de las cuestiones de evaluación}

A continuación resolveremos las cuestiones de evaluación planteadas.

\subsection{Cuestión 1}

Una vez definido las funciones de coste y los límites de producción de energía, el problema de optimización queda de la siguiente manera:

\begin{align*}
\min & 0.0578082\log(x_1^{100}) + 0.0016517x_2^2 + 0.0412916x_3\\
     & x_1 + x_2 + x_3 - 1000 = 0 \\
     & 1 \leq x_1 \leq 700 \\
     & 0 \leq x_2 \leq 500 \\
     & 0 \leq x_3 \leq 500
\end{align*}

\subsection{Cuestión 2}

Modificando de forma adecuada los valores de E1, E2, E3 y del vector ub, y resolviendo con Scilab tenemos que la solución del problema es la siguiente:

\begin{verbatim}
El coste óptimo del GW/hora (en miles de euros) es:   
 
    50.000033  
 
 Los MW/hora a producir por el primer calentador son:   
 
    699.99988  
 
 Los MW/hora a producir por el segundo calentador son:   
 
    12.499822  
 
 Los MW/hora a producir por el tercer calentador son:   
 
    287.5003  
\end{verbatim}

\subsection{Cuestión 3}

A tenor de los resultados devueltos por Scilab, tenemos que que el coste del GW/hora es de 50\,000 euros. Se deben producir 700 MW/hora del primer calentador, 12.5 MW/hora del segundo y 287.5 del tercer calentador.\\



\section{Conclusiones}



\end{document}